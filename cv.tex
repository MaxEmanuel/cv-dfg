\documentclass[checked=true,walterbenjamin=false]{dfg-cv-en}

%% Nachfolgend werden die Inhalte der ersten zwei Tabellen gesetzt

% Titel
\actitle{Dr.-Ing.\,habil.} 

% Vorname
\vorname{Alva Jona}

% Name 
\nachname{Müller} 

% Aktuelle Position --  (ggf. einschließlich Ende der Vertragslaufzeit)
\position{Akademische Oberrätin auf Lebenszeit} 

% Aktuelle Institution(en)/Ort(e), Land
\institution{Lehrstuhl für XYZ an der ABC Universität Musterstadt, Deutschland} 

% Identifikatoren/ORCID -- Antragstellende, die über eine ORCID-ID verfügen, sind aufgefordert, diese anzugeben. Antragstellende, die über keine ORCID-ID verfügen, sind eingeladen, aber nicht verpflichtet, eine solche zu erstellen.
\orcid{0000-0000-0000-0000} 

% Schule, Land -- Obligatorisch nur für das Walter-Benjamin-Programm; Zeitraum (monatsgetreu ab 1. Klasse) 
\schule{
    \begin{tabularx}{\linewidth}{@{}X}
        9/1980--7/1984: Grundschule Ort, Deutschland\\
        9/1984--7/1993: Gymnasium Ort, Deutschland
    \end{tabularx}
}



% Studium -- Fach, Zeitraum, Ort, Land (ggf. auch Wechsel im Fachgebiet)
\studium{
    \begin{tabularx}{\linewidth}{@{}X}
        Technische Informatik, 1994--2000, Musterstadt, Deutschland
    \end{tabularx}
}

% Promotion -- Datum, Betreuende/Mentorinnen/Mentoren, Fach (Angabe Fach ist optional), Einrichtung(en), Land
\promotion{31.07.2005, Prof.\ Dr.-Ing.\ Vorname Nachname, ABC Universität, Deutschland}

% Stationen des wissenschaftlichen/beruflichen Werdegangs (optionale Angaben seit der Promotion) -- Für den Antrag relevante Tätigkeiten sind chronologisch (die aktuellste am Anfang) mit der Angabe von Zeitraum, Station/Position und Einrichtung zu nennen, wie z. B. Forschungsaufenthalte, Habilitation (Thema/Fach, Betreuende), etc.
\stationen{
    \begin{tabularx}{\linewidth}{@{}X}
        2015--dato: Akademische Oberrätin auf Lebenszeit, Lehrstuhl für XYZ an der ABC Universität\\
        2011--2015: Akademische Rätin auf Lebenszeit, Lehrstuhl für XYZ an der ABC Universität\\
        2005--2011: Akademische Rätin auf Zeit, Lehrstuhl für ZYX an der DEF Univeristät Musterstadt, Deutschland\\
        2005--2010: Habilitation (Informatik, Prof.\ Dr.-Ing.\ Vorname Nachname)\\
    \end{tabularx}
}





\begin{document}

% --- Ergänzende Angaben zum Werdegang
% Nachfolgend können Sie freiwillig ergänzende Informationen zu Ihrem Werdegang oder einer besonderen persönlichen Situation eintragen, sollten Sie den Eindruck haben, dass diese Angaben für die angemessene Begutachtung und Bewertung Ihrer wissenschaftlichen Leistung relevant sein können.
\secWerdegang \optionaltext % optional, Freitextfeld (Zeile auskommentieren, wenn nicht genutzt)

\noindent Beispiele:
\begin{compactItemize}
    \item \enquote{first generation academic}
    \item Pflege von Angehörigen
    \item Chronische Erkrankungen
\end{compactItemize}


% --- Engagement im Wissenschaftssystem
% Nachfolgend können Sie Angaben zu weiteren Tätigkeiten im Wissenschaftssystem machen. Dazu zählen beispielsweise Gremientätigkeiten, Tätigkeiten in der Selbstverwaltung der Wissenschaft, die Organisation wissenschaftlicher Veranstaltungen, Aktivitäten in der Lehre sowie Tätigkeiten als Mentorin bzw. Mentor.
\secEngagement \optionaltext % optional, Freitextfeld (Zeile auskommentieren, wenn nicht genutzt)

\noindent Beispiele:
\begin{compactItemize}
    \item seit 2015: Mitglied der Studienkommission StuKo der ...
    \item Mitglied in Programmkommittees (z.\,B.\ ABC, DEF, HIJ)
\end{compactItemize}


% --- Betreuung von Forschenden in frühen Karrierephasen
\secBetreuung \semioptionaltext % teilweise optional, Freitextfeld -- für Graduiertenkollegs obligatorisch (Zeile auskommentieren, wenn nicht genutzt)

\noindent Beispiele:
\subsubsection*{Abgeschlossene Promotionen}

\begin{compactItemize}
\item 2014-2021: Vorname2 Nachname2, Thema1 (danach: PostDoc an der XYZ Uni)
\item 2017-2020: Vorname3 Nachname3, Thema2 (danach: Firma1)
\end{compactItemize}

\subsubsection*{Laufende Promotionen}
\begin{compactItemize}
\item seit 2020: Vorname4 Nachname4, Thema3    
\end{compactItemize}


% --- Wissenschaftliche Ergebnisse
% Nachfolgend sind Ihre wichtigsten öffentlich gemachten wissenschaftlichen Ergebnisse anzugeben
\secErgebnisse % Teil A obligatorisch, Teil B optional, Freitextfelder

% ---- Kategorie A
% In dieser Kategorie geben Sie bitte Fachaufsätze in Peer Review-Zeitschriften, Beiträge zu Konferenzen oder Sammelbänden jeweils mit Peer Review sowie Buchpublikationen an (siehe auch DFG-Vordruck 1.91). Es können maximal zehn Publikationen aufgelistet werden.
\secKatA % obligatorisch, Freitextfeld

\begin{compactEnumerate}
    \item Co-Autoren. \enquote{Papiertitel}. In: \emph{Konferenz}. Jahr, pp. XXX–XXX.
    \item Co-Autoren. \enquote{Papiertitel}. In: \emph{Konferenz}. Jahr, pp. XXX–XXX.
    \item Co-Autoren. \enquote{Papiertitel}. In: \emph{Konferenz}. Jahr, pp. XXX–XXX.
    \item Co-Autoren. \enquote{Papiertitel}. In: \emph{Konferenz}. Jahr, pp. XXX–XXX.
    \item Co-Autoren. \enquote{Papiertitel}. In: \emph{Konferenz}. Jahr, pp. XXX–XXX.
    \item Co-Autoren. \enquote{Papiertitel}. In: \emph{Konferenz}. Jahr, pp. XXX–XXX.
    \item Co-Autoren. \enquote{Papiertitel}. In: \emph{Konferenz}. Jahr, pp. XXX–XXX.
    \item Co-Autoren. \enquote{Papiertitel}. In: \emph{Konferenz}. Jahr, pp. XXX–XXX.
    \item Co-Autoren. \enquote{Papiertitel}. In: \emph{Konferenz}. Jahr, pp. XXX–XXX.
    \item Co-Autoren. \enquote{Papiertitel}. In: \emph{Konferenz}. Jahr, pp. XXX–XXX.
\end{compactEnumerate}

% ---- Kategorie B
% Nachfolgend können Sie die in jeder weiteren Form öffentlich gemachten Ergebnisse aus Ihrer Forschung anführen.
\secKatB \optionaltext % optional, Freitextfeld (Zeile auskommentieren, wenn nicht genutzt)

% --- Anerkennung durch das Wissenschaftssystem
% Nachfolgend können Sie Angaben zu Auszeichnungen oder Preisen machen. Dazu zählen auch Einladungen oder Berufungen in herausgehobenen Gremien oder Akademien.
\secAnerkennung \optionaltext % optional, Freitextfeld (Zeile auskommentieren, wenn nicht genutzt)

\noindent Beispiel:
\begin{compactItemize}
\item 2020: Konferenz, X Paper Award
\item 2013: Promotionspreis (EUR X) der Y, Ort1
\item 2010: Konferenz, X Paper Award
\end{compactItemize}


% --- Sonstige Angaben
% Nachfolgend können Sie auf weitere Punkte zur Charakterisierung Ihrer Person als Wissenschaftlerin bzw. Wissenschaftler oder auf andere Aspekte, wie beispielsweise eine Dual-Career-Thematik (die z.  B. die Standortwahl bedingt), hinweisen, die aus Ihrer Sicht relevant für die Begutachtung oder Bewertung des Antrags sind.
\secSonstiges \optionaltext % optional, Freitextfeld (Zeile auskommentieren, wenn nicht genutzt)

\end{document}
